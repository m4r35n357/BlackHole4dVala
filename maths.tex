\documentclass[11pt]{article}
%Gummi|063|=)
\usepackage{datetime}
\usepackage{mathtools}
\usepackage{listings}
\usepackage{graphicx}
\title{\textbf{Simulating Geodesics in the Kerr Spacetime}}
\author{Ian Smith}
\date{May 2016}
\graphicspath{ {images/} }
\begin{document}

\maketitle

\abstract
Here is a simple and accurate numerical approach for efficiently generating 4D geodesics for particles and photons in the Kerr-deSitter spacetime.
Simplicity is achieved by working with the constants of motion for the Kerr spacetime rather than solving the geodesic equations directly, so that the equations of motion are first-order differential equations rather than second-order.
Particular attention is paid to techniques for generating a wide variety of useful initial conditions for the simulator, both for particles and light pulses.
There are two integration methods within the simulator; one using Runge-Kutta (RK4) integrators, and one using symplectic integration.
Both integrators use a fixed time-step.
The parameter generation and simulation core is written in Vala, whilst graphical output and processing, including error plots, are provided by Python modules and shell scripts.
There is also a wxMaxima file for generating and checking the equations used in this article.
The source code for the suite of programs is publicly available under a BSD licence.

\section{Bound Orbits in the Kerr Spacetime}

The starting point for the simulation procedure is the Kerr line element, expressed in terms of the widely used Boyer-Lindquist coordinates.
\begin{equation}
\begin{aligned}
d s^2 &= \frac{ \sin^2 \theta } {r^2 + a^2 \cos^2 \theta} (a dt - (r^2 + a^2)d \phi )^2 + \frac{ r^2 + a^2 \cos^2 \theta } {(r^2 - 2Mr  + a^2)} d r^2 \\
&+ (r^2 + a^2 \cos^2 \theta) d \theta^2 + \frac{ (r^2 - 2Mr  + a^2) } {r^2 + a^2 \cos^2 \theta} (dt - a \sin^2 \theta d \phi )^2
\end{aligned}
\end{equation}
From a line element there is a well-defined but complicated process to generate the Christoffel Symbols, form the geodesic equations, and solve them.
The geodesic equations corresponding to the Kerr spacetime are complex second order differential equations which are well known and can be solved using standard numerical integration techniques, typically one of the Runge-Kutta methods.

Rather than attempting to formulate and solve these geodesic equations directly, the approach taken here is to evolve the first order equations of motion expressed in terms of the constants of motion $E$ (energy), $L$ (equatorial angular momentum), and $Q$ (Carter's constant).
The approach is so straightforward that the RK4 version of the simulator script consists of just over a hundred lines of Python code, and well under 200 in Vala \cite{m4r35n357}.

\subsection {Equations of motion}

Here is the well established set of first order equations describing Kerr orbits in terms of $E$, $L$, and $Q$:
\begin{align}
\frac{d t}{d \lambda} &= \frac{(r^2 + a^2) \left((r^2 + a^2) E - aL \right)} {(r^2 - 2Mr  + a^2)} + a(L - aE \sin^2 \theta) \\
\frac{d r}{d \lambda} &= \pm \sqrt {R(r)} \\
\frac{d \theta}{d \lambda} &= \pm \sqrt {\Theta (\theta)} \\
\frac{d \phi}{d \lambda} &= \frac{a \left((r^2 + a^2) E - aL \right)} {(r^2 - 2Mr  + a^2)} + \left(\frac {L} {\sin^2 \theta} -aE \right)
\end{align}
where
\begin{equation}
d \lambda = \frac {d \tau} {(r^2 + a^2 \cos^2\theta)}
\end{equation}
These are essentially equations 2 and 3 from \cite{wilkins}, fully expanded except for the two potential functions $R$ and $\Theta$ which, together with their differentials, are defined in equations 7-10 below.

Notice also the use of the "Carter-Mino" time variable, $\lambda$, to render the $R$ and $\Theta$ equations (plus their differentials of course) mutually independent (essential for the symplectic integrators, but not for the RK4 integrator).
In terms of this time variable, the potentials are simply the squares of the $r$ and $\theta$ velocities.
\begin{align}
   \begin{split}
    R(r) &= \left((r^2 + a^2) E - aL \right)^2 \label{eq:1}\\
    &- (r^2 - 2Mr  + a^2) \left(Q + ( L - aE)^2 + \mu^2 r^2 \right)
   \end{split}\\
   \begin{split}
    \frac{d R}{d r} (r) &= 4rE \left((r^2 + a^2)E - aL \right) \label{eq:2}\\
    &- 2\mu^2r(r^2 - 2Mr  + a^2)\\
    &- (2r - 2M) \left(Q + ( L - aE)^2 + \mu^2 r^2 \right)
   \end{split}\\
  \Theta (\theta) &= Q - {\cos^2 \theta } \left( \frac{L^2}{\sin^2 \theta } + a^2( \mu^2 - E^2) \right) \label{eq:3}\\
  \frac{d \Theta}{d \theta} (\theta) &= 2 \cos \theta \sin \theta \left(\frac{L^2} {\sin^2 \theta } + a^2(\mu^2 - E^2) \right) +\frac{2 L^2 \cos^3 \theta } {\sin^3 \theta } \label{eq:4}
\end{align}

The $t$ and $\phi$ equations (1 and 4) can be evolved trivially using RK4, but the $R$ and $\Theta$ equations (2 and 3) require a little more effort.
Because of the square roots of the potential functions, the RK4 integration approach is made more difficult by the need to identify turning points, which is troublesome to achieve reliably, and even harder to do without compromising the accuracy of the simulation.
I have therefore implemented two integrators here.
One is an RK4 integrator, with a simple empirical solution to the turning point issue, and the other is a symplectic integrator which avoids the turning point issue entirely, at the cost of needing derivatives of the potentials.

The key to the symplectic approach is to square the $R$ and $\Theta$ equations, and use a Stormer-Verlet based integrator \cite{hairer} to evolve the resulting equations, which are each treated as a "pseudo hamiltonian" system of first order differential equations.  This turns out to be a surprisingly simple solution, and is possible because the potential equations (when squared) are both of the form:

\begin{equation}
\frac {\dot x^2 - V(x)}{2} = 0
\end{equation}

This expression can be treated numerically in the same way as a regular hamiltonian (there is no mass in the "kinetic" term so the velocity is also the canonical momentum), in particular it is amenable to symplectic integration \cite{hairer}.
It is also used to quantify the integration errors for $r$ and $\theta$.
The two differentiated potentials in equations 7 and 9 are used for velocity/momentum updates in the integrator routines.

The symplectic uses composition \cite{hairer} to enable changing the (even) integration order from 2 in the case of basic Stormer-Verlet, up to a maximum of 6th order.
Rather than using Euler or RK4, the $t$ and $\phi$ equations are actually processed alongside the composed Stormer-Verlet updates, empirically increasing the order and accuracy for $t$ and $\phi$ (according to the four-velocity norm error).

\section{Finding initial conditions}

Another practical difficulty in generating orbits is the problem of generating suitable sets of initial conditions. Finding an interesting set by trial and error is hard, partly because many combinations of the constants of motion are unphysical, and also because of extreme sensitivity to some parameters (particularly $E$).
In the case of light there already exist closed form expressions for $L$ and $Q$ ($E$ is generally taken to be 1) in terms of $r$ and $a$ in \cite{teo}

Here I present a straightforward way of generating two types of bound orbits for particles in three spatial dimensions.
It is based on solving sets of three potential equations under various conditions using the three constants of motion $E$, $L$, and $Q$ as variables, and turning points of the potentials (with respect to these variables) as constraint parameters.

\subsection{Constant radius ("spherical") orbits}

For constant radius orbits at $r_0$, we are looking for a double root (local maximum) of the quartic $R(r)$, in other words we want the radial velocity to be zero, and its differential to also be zero so that the radial speed remains zero during the orbit.
$\Theta (\theta)$ will also be zero at the maximum deviation from equatorial (the minimum value of $\theta$).

\begin{align}
R(r_0, E, L, Q) &= 0 \\
\frac{d R}{d r} (r_0, E, L, Q) &= 0 \\
\Theta(\theta_{MIN}, E, L, Q) &= 0
\end{align}

\subsection{Variable radius ("spherical shell") orbits}

For variable radius orbits we are looking for two distinct roots of the quartic $R(r)$, in other words that the orbit is bound between two $r$ values, $r_1$ and $r_2$.
The $\Theta (\theta)$ condition is unchanged from the constant radius case above.

\begin{align}
R(r_1, E, L, Q) &= 0 \\
R(r_2, E, L, Q) &= 0 \\
\Theta(\theta_{MIN}, E, L, Q) &= 0
\end{align}

\subsection{Plummeting (an aside)}

These "orbits" are not the focus of this article, but it is possible to experiment by setting $L < 1.0$ after generating a spherical orbit as in equations 11-13 above.

\subsection{Finding the roots}

Equations 11-13 and 14-16 can of course be solved by various root-finding techniques, the programs described here use the "MultirootFsolver" nonlinear solvers from the Gnu Scientific Library (GSL).

A typical invocation of the suite of programs might be:
\begin{verbatim}
./icgen dnewton 3.0 12.0 0.15 1.0 1.0 >$ic 2>$pot
\end{verbatim}
(where \$ic and \$pot represent file locations for the initial conditions and potential plot data) which generates an orbit bound between 3 and 12 radial units and with a minimum latitude of 0.15 radians (starting from the "north pole" of course).  In this example the spin parameter is 1.0, whilst the final parameter is a scaling factor for the computed angular momentum (this is normally left at 1.0; smaller values allow generation of plummeting trajectories).

Once the solver has terminated successfully it writes an initial conditions file (including the final values of $E$, $L$, and $Q$ as well as optimizer status) in JSON format for use as input to the simulator.  Here is an example corresponding to the command above:
\begin{verbatim}
{ "solver" : "dnewton",
  "iterations" : 7,
  "errors" : "-1.421e-14 0.000e+00 0.000e+00",
  "M" : 1.0,
  "a" : 1.0,
  "mu" : 1.0,
  "E" : 0.94243182428522676,
  "Lz" : 1.3515589497079128,
  "C" : 7.124975646505507,
  "r" : 7.5,
  "theta" : 1.570796327,
  "start" : 0.0,
  "duration" : 5000.0,
  "step" : 0.001,
  "plotratio" : 500,
  "integrator" : "sc4"
}
\end{verbatim}
The simulator is invoked like so:
\begin{verbatim}
./bh3d.rk4 <$ic
\end{verbatim}
The simulator takes this input, calculates the geodesic and writes its output data to standard output in JSON format (which can be redirected to a file or piped to another program for display or other purposes).  A static image of the corresponding geodesic (displayed using a VPython script) is shown below.  VPython allows the user to zoom and rotate the view, and watch it evolve as it is being plotted.
\begin{figure}[h]
\includegraphics[width=\textwidth]{Screenshot}
\end{figure}
The output of the simulator has been informally checked against a number of publicly available result sets and programs to guard against the possibility of gross errors.  This includes  but is not limited to the program GRorbits for equatorial geodesics \cite{grorbits}, together with published papers for spherical particle orbits \cite{teo}, and photon orbits \cite{kheng}.

\subsection{. . . but are they REALLY geodesics?}

Despite the high level of accuracy which the RK4 and symplectic integrators can achieve for $r$ and $\theta$, this provides no guarantee that the complete solution is an accurate geodesic.  To verify this, we can evaluate the four-velocity norm of the path by plugging all the coordinate derivatives into the line element.  This error data is part of the JSON output of the simulation.  The figure below shows the error plot for the simulation above: blue is the radial error, red the latitudinal, and green the (deviation from) four-velocity norm, all plotted against a false dB scale (where 0dB = 1.0).
\begin{figure}[h]
\includegraphics[width=\textwidth]{figure_1}
\end{figure}
This type of error plot makes it easy to see the effects on accuracy of changing the timestep and integrator order; the default values of 2nd order integration and a fairly course timestep are set that way to encourage experimentation.

\section{The programs}

The sources for the programs \cite{m4r35n357} described here are publicly available on GitHub, under a BSD licence.  These are part of a suite of very small scripts and programs which are intended to implement the methods in this article as concisely as possible.  There is a short README file in the root directory which gives basic instructions for running the simulations.

\begin{thebibliography}{1}
\bibitem{wilkins} Wilkins, D.C., "Bound Geodesics in the Kerr Metric"
\bibitem{hairer}  Hairer, E., Hairer, M., "GniCodes - Matlab programs for geometric numerical integration "
\bibitem{kheng} Lim Yen Kheng, Seah Chu Perng, Tan Boon Sze Jackson, "Massive Particle Orbits Around Kerr Black Holes"
\bibitem{teo} Teo, E, "Spherical photon orbits around a Kerr black hole"
\bibitem{grorbits} Tujela, S., "GRorbits, http://stuleja.org/grorbits/"
\bibitem{m4r35n357} Smith, I. C., "4D Kerr Spacetime Geodesic calculation and display, https://github.com/m4r35n357/BlackHole4DVala.git"
\end{thebibliography}

\end{document}

